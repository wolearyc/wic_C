\#\-W\-I\-C\-K G\-U\-I\-D\-E (R\-E\-A\-D T\-H\-I\-S!)\# Wick is fully documented, but some elements of library do need some explaining. Here are some things you should know about wick before you use it. \#\-A. Struct Overview\#

Wick uses structs to store information. Wick's structs represent \char`\"{}game objects\char`\"{}, such as Images and Text. There are not very many of them, so I can give you an quick overview.

{\bfseries \hyperlink{struct_game}{Game}} One instance of \hyperlink{struct_game}{Game} should exist throughout the life of the entire program. As the name suggests, it represents the entire \hyperlink{struct_game}{Game}.

{\bfseries \hyperlink{struct_pair}{Pair}} A \hyperlink{struct_pair}{Pair} is simply a \hyperlink{struct_pair}{Pair} of double precision floating point numbers. Pairs represent coordinates, 2\-D vectors, 2\-D dimensions, and basically everything 2\-D.

{\bfseries \hyperlink{struct_color}{Color}} \hyperlink{struct_color}{Color} represents an R\-G\-B\-A color.

{\bfseries \hyperlink{struct_polygon}{Polygon}} A filled \hyperlink{struct_polygon}{Polygon} which can be drawn to the screen. A \hyperlink{struct_polygon}{Polygon} can be moved, scaled, rotated, and colored.

{\bfseries \hyperlink{struct_quad}{Quad}} A filled rectangle which can be drawn to the screen. A \hyperlink{struct_quad}{Quad} can be moved, scaled, rotated, and colored.

{\bfseries \hyperlink{struct_texture}{Texture}} A \hyperlink{struct_texture}{Texture} is a texture loaded from either the file system of some buffer of unsigned chars.

{\bfseries \hyperlink{struct_image}{Image}} An \hyperlink{struct_image}{Image} can be drawn to the screen. It derives its data from some \hyperlink{struct_texture}{Texture}. Images can be moved, scaled, rotated, and colored.

{\bfseries \hyperlink{struct_bounds}{Bounds}} A \hyperlink{struct_bounds}{Bounds} is a bounding rectangle. It is usually used to define what part of a \hyperlink{struct_texture}{Texture} and \hyperlink{struct_image}{Image} should use for drawing.

{\bfseries Font} A Font at a specific point size loaded from a file.

{\bfseries Text} A Text can be drawn to the screen. It derives its data from some string and a Font. A Text can be moved, scaled, rotated, and colored.

\#\-B. Struct Initialization\# \hyperlink{struct_pair}{Pair} and \hyperlink{struct_bounds}{Bounds} should be initialized manually. {\ttfamily \hyperlink{struct_pair}{Pair} a = \{0,0\};} {\ttfamily \hyperlink{struct_bounds}{Bounds} b = \{(\hyperlink{struct_pair}{Pair}) \{0,0\}, (\hyperlink{struct_pair}{Pair}) \{64,64)\}\};}

However, most structs really shouldn't be initialized manually. These structs use an \char`\"{}init\char`\"{} function. An init function always has the name init\-\_\-structname, where struct name is the all-\/lowercase name of the struct.

Thus, to initialize a color, for example\-: \hyperlink{struct_color}{Color} target; {\ttfamily init\-\_\-color(\&target, 0, 0, 0, 255);}

The use of an init function ensures that more complex structs are initialized completely and correctly, and also allows for error reporting (which we'll discuss after another section).

\#\-C. Memory management\# Some structs need to be freed once you're done with them. With Structs that need to be freed, you'll see a \char`\"{}free\char`\"{} function that coorosponds to the struct. Similar to the init function, free functions are named free\-\_\-structname, which struct name is the all-\/lowercase name of the struct. For example\-:

{\ttfamily \hyperlink{struct_texture}{Texture} texture;} {\ttfamily /$\ast$ initialize texture via init\-\_\-texture} {\ttfamily /$\ast$ do stuff involving texture} {\ttfamily free\-\_\-texture(\&texture);}

In this example, texture is allocated on the stack. Once free\-\_\-texture is called, texture is still on the stack, but all the heap space given to elements of texture is gone.

Alternatively, one can store a \hyperlink{struct_texture}{Texture} (or any other struct) on the heap\-:

{\ttfamily Texture$\ast$ texture = malloc(sizeof(\-Texture));} {\ttfamily /$\ast$ initialize texture via init\-\_\-texture} {\ttfamily /$\ast$ do stuff involving texture} {\ttfamily free\-\_\-texture(texture);} {\ttfamily free(texture);}

free\-\_\-texture still only deallocates the heap space given to {\itshape elements} of texture. A final \char`\"{}free\char`\"{} is required to completely deallocate the texture.

\#\-D. Modifying structs\# Most elements in a given struct can be modified directly. For example the following is totally safe\-:

{\ttfamily \hyperlink{struct_image}{Image} image;} {\ttfamily /$\ast$ initialize image via init\-\_\-image} {\ttfamily image.\-location = (\hyperlink{struct_pair}{Pair}) \{50,50\};}

However, some elements (which are undocumented) should not be modifed directly. The name of these elements starts with \char`\"{}p\-\_\-\char`\"{}, the p standing for \char`\"{}private\char`\"{}. These elements either should never be modifed, or can be modified correctly only through a function. For example, the the p\-\_\-dimensions element in \hyperlink{struct_quad}{Quad} should not be modified directly. Instead, the following function should be used instead\-:

{\ttfamily int \hyperlink{quad_8h_a4564577609fd9008e7883db2fb210109}{set\-\_\-quad\-\_\-dimensions(\-Quad$\ast$ target, Pair dimensions)}}

\#\-E. Error reporting\# Many functions in wick return integer \char`\"{}error codes\char`\"{}. Such functions return 0 on success and some code less than 0 on failure. These codes can be translated via\-:

{\ttfamily const char$\ast$ \hyperlink{error_8h_a8af34de9e5f9e832e17e37f775319796}{translate\-\_\-error\-\_\-code(int code)}}

If, for some reason, you don't have access to the error code, the last reported error code can be fetched via\-:

{\ttfamily int \hyperlink{error_8h_a92540a3cba575aec179ef8453725ecd1}{get\-\_\-last\-\_\-error()}}

If a function fails, returning a negative integer, wick guarantees (unless otherwise noted) that no variables, pointers, or anything will be modified.

\#\-F. Units\# Wick uses the same units across all structs and functions.

{\bfseries Screen coordinates} are measured in pixels from the lower left corner of the window.

{\bfseries Coordinates in general} are measured in pixels.

{\bfseries Rotation} is measured in radians from the positive x axis.

{\bfseries Scale values} are measured with Pairs, where each element in the \hyperlink{struct_pair}{Pair} is a multipier for the scale. For example, a scale value of(1,1) is original size, which a scale value of (0.\-5,0.\-5) is half size.

Some game objects, like Polygons and Quads, take vertices. These vertices are $\ast$$\ast$$\ast$relative to the object$\ast$$\ast$, meaning that they're measured in pixels relative to the object's location. 